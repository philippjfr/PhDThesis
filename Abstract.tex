\pdfbookmark[1]{Abstract}{Abstract} % Bookmark name visible in a PDF viewer

\begingroup
\let\clearpage\relax
\let\cleardoublepage\relax
\let\cleardoublepage\relax

\chapter*{Abstract} % Abstract name

How does the brain make use of information encoded in its circuits
during development? The sensory cortex has been known to extract low
level features of its visual environment through developmental and
activity-dependent processes. Various computational and theoretical
models exist describing various phenomena observed in the primary
visual cortex (V1) ranging from the development of orientation tuning,
anatomical organization of its circuits, surround modulation and
higher level effects like attention. However so far no model has been
able to bridge these levels of description.

In this thesis we integrate the information gleaned about cortical
development, circuit and cell-type specific interactions, anatomical
measurements as well as behavioural and electrophysiological
measurements to develop a computational model of V1 that is better
constrained and can make predictions about each of these levels of
evidence. Through a series of models, which incorporate increasing
levels of biophysical detail, and therefore are better constrained than
many previous models, we make detailed predictions on the types of
interactions required for robust development of cortical maps,
anatomical organization and computations performed by the primary
visual cortex. This also allows us to highlight wher.

Initial models will demonstrate how anatomical and
electrophysiological knowledge can be combined to give a unified
account of spatial tuning of the primary visual cortex. As a second
step we will investigate the known interactions between different
excitatory and inhibitory cell classes in V1 and what role each cell
type may play during development. Finally, we demonstrate how various
cell classes can interact to give rise to not only robust development
but also the development of realistic lateral connectivity patterns,
reflecting the statistics of the visual environment the model was
trained on. This allows us to explore how the model is able to capture
higher-order information about the environment and use that
information to optimize neural coding and aid the processing of even
complex visual tasks. 

In summary, the work presented herein will provide an account of the
computations performed by the primary visual cortex that is
constrained by a large body of evidence and previous modeling work and
sheds light on discrepancies between earlier results.

\endgroup
