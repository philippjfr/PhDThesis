\pdfbookmark[1]{Abstract}{Abstract} % Bookmark name visible in a PDF viewer

\begingroup
\let\clearpage\relax
\let\cleardoublepage\relax
\let\cleardoublepage\relax

\chapter*{Abstract} % Abstract name
How do circuits in the mammalian cerebral cortex encode properties of
the sensory environment in a way that can drive adaptive behavior?
This question is fundamental to neuroscience, but it has been very
difficult to approach directly.  Various computational and theoretical
models can explain a wide range of phenomena observed in the primary
visual cortex (V1), including the anatomical organization of its
circuits, the development of functional properties like orientation
tuning, and behavioural effects like surround modulation. However, so
far no model has been able to bridge these levels of description to
explain how the observed machinery leads to behavior.  Bridging these
levels is important, because phenomena at any one specific level can
have many possible explanations, but there are far fewer possibilities
to consider once all of the available evidence is taken into account.

In this thesis we integrate the information gleaned about cortical
development, circuit and cell-type specific interactions, and
anatomical, behavioural and electrophysiological measurements, to
develop a computational model of V1 that is constrained enough to make
predictions across multiple levels of description. Through a series of
models incorporating increasing levels of biophysical detail and
becoming increasingly better constrained, we are able to make detailed
predictions for the types of mechanistic interactions required for
robust development of cortical maps that have a realistic anatomical
organization, and thereby gain insight into the computations performed
by the primary visual cortex.

The initial models focus on how existing anatomical and
electrophysiological knowledge can be integrated into previously
abstract models to give a unified account of the emergence of
pattern-specific tuning in the primary visual cortex.  More detailed
models then address the interactions between specific excitatory and
inhibitory cell classes in V1, and what role each cell type may play
during development and function. Finally, we demonstrate how these
cell classes come together to form a circuit that gives rise not only
to robust development but also the development of realistic lateral
connectivity patterns.  Crucially, these patterns reflect the
statistics of the visual environment to which the model was exposed
during development. This property allows us to explore how the model
is able to capture higher-order information about the environment and
use that information to optimize neural coding and aid the processing
of complex visual tasks.

Using this model we can make a number of very specific predictions
about the mechanistic workings of the brain. Specifically the model
predicts a crucial role of parvalbumin-expressing interneurons in
robust development and divisive normalization, while it implicates
somatostatin immunoreactive neurons in mediating longer range and
feature-selective suppression. The model also highlights how these two
cell classes contribute to efficient neural coding and under what
conditions the model fails to organize, for example due to decoupling
of activity between the principal excitatory population and the
parvalbumin population, which may have implications for a variety of
diseases, where parvalbumin interneuron function is impaired such as
schizophrenia and autism. Further the model explains the switch from
facilitatory to suppressive surround modulation effects as a simple
byproduct of the facilitating response function of the somatostatin
population. Finally the model allows us to make predictions about the
statistics that are encoded in the extensive network of long-range
intra-areal connectivity in V1, suggesting that even V1 can capture
high-level statistical dependencies in the visual environment.

The final model represents a comprehensive and well constrained model
of the primary visual cortex, which for the first time can relate the
physiological properties of individual cell classes to their role in
development, learning and function. This work is also highly relevant
for clinical neuroscience, as the cell types studied here have been
implicated in neurological disorders as wide ranging as autism,
schizophrenia and Parkinson's.
\endgroup
