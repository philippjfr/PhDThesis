\pdfbookmark[1]{Lay Summary}{Lay Summary} % Bookmark name visible in a PDF viewer

\begingroup
\let\clearpage\relax
\let\cleardoublepage\relax
\let\cleardoublepage\relax

\chapter*{Lay Summary} % Abstract name
The field of neuroscience has made tremendous progress over the last
century.  Neuroscientists have collected an overwhelming amount of
data describing the brain at large and small levels, ranging from
interactions between individual genes, proteins and cells to the
communication between different brain regions. Yet we still understand
very little about how the brain actually implements the major
functions of a nervous system, such as learning about an animal's
environment to solve even basic sensory and cognitive tasks.  Fully
explaining such processing requires linking across all these levels of
description, explaining how the high-level behavior that can be
observed externally is constructed from the operation of the
lower-level elements that can be observed inside the brain.

Making such links is very difficult to do using laboratory
experiments, which for practical reasons almost always focus on one or
two levels at any one time.  Computational models can be an important
complementary tool, allowing researchers to express and then test
hypotheses about how these levels relate to each other.  Yet the vast
majority of existing models focus on a very narrow range of phenomena,
mirroring specific experiments rather than helping integrate
experimental results into a coherent explanation.  Unfortunately, any
specific experiment could have a huge range of possible explanations,
and only by considering all available evidence can we narrow in on the
correct explanations for how the brain functions.

In this thesis we present a computational model of visual cortex that
helps bridge across several disparate levels, using lower-level
elements constrained by measurements of different cell types and their
connections, grouped into neural regions that together form a sensory
subsystem, and then tested using sensory stimuli used in behavioral
experiments.  The resulting model is a good match to low-level
observations of neural connectivity, providing a specific and testable
explanation for how that connectivity arises, and is also closely
matched to the observed neural responses to a wide range of visual
patterns used in experiments.  The model demonstrates for the first
time how specific connections between neurons in the cortex can store
statistical information about the visual environment and then use that
information to optimize perception of novel stimuli.

This work is clearly important for basic neuroscience, which seeks to
understand how the brain functions.  But it is also important for
clinical neuroscience, because many neurological conditions have been
associated with various defects in the development and wiring of the
specific cell types studied here.  Future integrative models that can
link neural machinery to behavior will help drive progress in both of
these areas of research.
\endgroup
