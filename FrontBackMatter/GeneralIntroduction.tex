%!TEX root = ../PhDthesis.tex
\chapter{Introduction}

The overall aim of this thesis is to begin unifying across various
phenomenological models integrating them with existing mechanistic
models and come up with better constrained models of visuo-cortical
development reaching across all four levels of description. Starting
with developmental models of receptive field and orientation map
development, the models will work backwards from the lowest level
incorporating increasing levels of physical/anatomical detail,
hypothesizing about different algorithmic and computational roles for
different cell classes and analysing how they contribute both to the
self-organization and learning in the developing cortex but also their
potential role in the computations performed by the adult cortex. In
doing so we will also describe an incredibly flexible yet reproducible
workflow developed as part of this thesis to support the analysis of
complex, non-linear dynamic systems.

\section{Objectives}

One of the major experimental advances in visual neuroscience over the
past decade has been the ability to manipulate and record from
individual cells or cell classes by combining advanced genetic and
imaging techniques. As we will outline in the literature review this
has given us a new window into the mammalian brain and has allowed us
to begin hypothesizing what role different cell classes may play in
development and behavior. However, so far very few attempts have been
made to place these new findings into a greater context, particularly
to investigate what makes different cell classes suited toward
particular functions.

In particular we will adapt an existing developmental model of the
primary visual cortex, which gives a good account of the
activity-dependent development in the primary visual cortex,
demonstrating robust, yet adaptive organization of receptive fields
and lateral connectivity. Once we have established how to relate the
spatial scales in the model to experimental data this thesis sets out
to answer a number of questions:

\begin{itemize}
\item What type of inhibition is suited toward driving robust
  self-organization of V1 circuits.
\item What is the role of parvalbumin- and somatostatin-immunoreactive
  interneurons in the development and function of the primary visual cortex?
\item What properties make a particular cell class suited for a
  particular function?
\item How do the cell classes self-organize into a circuit that is
  robust and stable to the widely varying visual input that an animal's
  environment provides?
\item What specific computational purpose, if any, do particular cell
  classes serve?
\item Does the connectivity in V1 store the visual statistics of the
  environment and how can it begin using this information to improve
  neural coding and low-level inference?
\end{itemize}

Through a detailed breakdown of the literature we show how the spatial
scales of the circuitry of different cell classes in the visual cortex
can form an effective mexican-hat driving the self-organization of
feature maps. Having established a clear mapping between anatomical
and physiological measurements and the model we investigate the how
the response properties of different cell classes bring about the
known spatial profiles making several predictions. The fast, quickly
spiking response properties of parvalbumin-expressing neurons lead to
a broad orientation tuning, which is crucial for robust development of
cortical feature maps. The facilitating response properties of
somatostatin immunoreactive neurons on the other hand lead to narrower
orientation tuning, allowing these neurons to integrate over large
spatial scales mediating orientation dependent suppression. Finally we
demonstrate how long-range lateral connections can capture the visual
statistics of the environement.

The guiding hypothesis of this thesis is the idea that the incredible
complexity of the mammalian cortex is achieved through a careful
genetic program that establishes a basic circuit, which through the
process of development self-organizes into a circuit, which can
robustly capture the statistics of the input and adaptively use this
information to optimally encode this input for further processing and
perform low-level inferences when only a low amount of information is
available.

\section{Organisation of the Thesis}

Based on the overarching goals laid out above it is crucial to begin
by integrating information across Marr's levels of description. The
entire first chapter will therefore be devoted towards providing an
overview of our current knowledge about the visual system, broadly
categorized into the four levels of analysis.

In the second results chapter we extend this analysis to models of
cortical development and function that takes into account the
anatomical and physiological properties of different cell classes and
begin elucidating their role in experimentally observed phenomena
ranging from self-organization to particular computations performed by
the fully developed cortex, including surround modulation, divisive
gain control and sparse coding.

In the final results chapter we demonstrate how a model that respects
the known properties of different cell classes and their interactions
can develop realistic connectivity patterns that not only allow for
the resulting circuit to robustly respond to highly variable input but
also encode the statistics of the input to aid in the various complex
computations performed by the cortex.

By the end of the thesis we will have presented a unifying account on
how different cell classes in the visual cortex self-organize into a
robust circuit that can encode the natural image statistics of its
rearing environment.
