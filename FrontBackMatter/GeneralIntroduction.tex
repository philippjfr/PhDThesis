%!TEX root = ../PhDthesis.tex
\chapter{Introduction}

In introducing the topic of neuroscience it is often customary to
point out the incredibly complexity of the mammalian nervous system
and highlight its incredible ability to adapt to whatever environment
it develops in. While this serves to highlight just how interesting a
topic it is to begin revealing the computations performed by this
structure, it also implicitly highlights the difficulty of the
problem. Particularly when trying to develop computational models of
the nervous system it is often tempting to abstract away the
complexities of biophysical reality and focus on developing toy models
that describe particular phenomena very well but lack the ability to
provide more general predictions.

This is precisely what David Marr highlighted when stating that the
brain and the visual system in particular should be analyzed at three
distinct but complementary levels of analysis, later expanded to
include a fourth level by Tomaso Poggio. According to Marr and Poggio
models of cognitive systems such as the mammalian visual system should
describe the system at one or more of the following levels of
analysis:

\begin{itemize}
\item Learning: How does the system learn to perform the necessary computations?
\item Computational: What does the system actually do? What computations does it perform?
\item Algorithmic/Representational: How does the system represent the problem and solve it?
\item Implementational/Physical: How is the system physically implemented?
\end{itemize}

Each of these levels of evidence when viewed in isolation can
contribute to the overall understanding about the system, however to
gain real insights, models should begin abstracting across these
levels of analysis to find out where and how evidence from various
sources fit together or more interestingly, highlight potential
discrepancies.

Visual neuroscience in particular has enjoyed a huge amount of
attention with a wide range of descriptive and mechanistic models
describing individual phenomena very well. There is however still a
serious lack of models that attempt to bridge between our
computational understanding of what the visual cortex is doing and how
it self-organizes with the lower level details of how the physical
substrate could actually perform those computations.

The aim of this thesis is to begin unifying across various
phenomenological models integrating them with existing mechanistic
models and come up with better constrained models of visuo-cortical
development reaching across all four levels of description. Starting
with developmental models of receptive field and orientation map
development, the models will work backwards from the lowest level
incorporating increasing levels of physical/anatomical detail,
hypothesizing about different algorithmic and computational roles for
different cell classes and analysing how they contribute both to the
self-organization and learning in the developing cortex but also their
potential role in the computations performed by the adult cortex. In
doing so we will also describe an incredibly flexible yet reproducible
workflow developed as part of this thesis to support the analysis of
complex, non-linear dynamic systems.

\section{Organisation of the Thesis}

Based on the overarching goals laid out above it is crucial to begin
by integrating information across Marr's levels of description. The
entire first chapter will therefore be devoted towards providing an
overview of our current knowledge about the visual system, broadly
categorized into the four levels of analysis.

Having established the current state of knowledge about the visual
system we then explore how providing a unified account of the spatial
scales in anatomical, electrophysiological and behavioural experiments
allows us, for the first time to bridge across all three levels and
begin highlighting where these very different experiments make similar
predictions and where there are seeming discrepancies and highlight
where these differences could arise from.

In the second results chapter we extend this analysis to models of
cortical development and function that takes into account the
anatomical and physiological properties of different cell classes and
begin elucidating their role in experimentally observed phenomena
ranging from self-organization to particular computations performed by
the fully developed cortex, including surround modulation, divisive
gain control and sparse coding.

In the final results chapter we demonstrate how a model that respects
the known properties of different cell classes and their interactions
can develop realistic connectivity patterns that not only allow for
the resulting circuit to robustly respond to highly variable input but
also encode the statistics of the input to aid in the various complex
computations performed by the cortex.

By the end of the thesis we will have presented a unifying account on
how different cell classes in the visual cortex self-organize into a
robust circuit that can encode the natural image statistics of its
rearing environment and meaningfully contribute to incredibly complex
computations and inferences performed by ``most complex known
structure in the universe''.
