%!TEX root = ../PhDthesis.tex
\chapter{Introduction}

One of the major experimental advances in visual neuroscience over the
past decade has been the ability to manipulate and record from
individual cells or cell classes by combining advanced genetic and
imaging techniques. This has given us a new window into the mammalian
brain and has allowed us to begin hypothesizing what role different
cell classes may play in development and behavior. However, so far
very few attempts have been made to place these new findings into a
greater context, particularly to investigate what makes different cell
classes suited toward particular functions and how these neurons
develop into a functional circuit, which can efficiently extract and
encode the statistics of the visual input.

The guiding hypothesis of this thesis is the idea that the incredible
complexity of the mammalian cortex is arrived at through a process of
self-organization giving rise to a developed circuit, which can
robustly capture the statistics of the input and adaptively use this
information to optimally encode the input for further processing and
perform low-level inferences when only a low amount of information is
available. Over the course of this thesis I will outline how a fairly
stereotyped circuit of different neural cell classes develops into a
circuit that closely matches experimental observations of development
and surround modulation. This will for the first time bridge the early
development of the circuits, with a mechanistic account of how the
learned statistics are used and finally play a role in specific
cortical functions, which give rise to contextual modulation effects
and higher level behaviors.

In particular I will adapt an existing developmental model of the
cortical development, which gives a good account of the
activity-dependent development in the primary visual cortex,
demonstrating robust, yet adaptive organization of receptive fields
and lateral connectivity. Once I have established how to relate the
spatial scales in the model to experimental data this thesis sets out
to answer a number of questions:

\begin{itemize}
\item What type of inhibition is suited toward driving robust
  self-organization of V1 circuits?
\item What is the role of parvalbumin- and somatostatin-immunoreactive
  interneurons in the development and function of the primary visual
  cortex?
\item What properties make a particular cell class suited for a
  particular function?
\item How do the cell classes self-organize into a circuit that is
  robust and stable to the widely varying visual input that an
  animal's environment provides?
\item What specific computational purpose, if any, do particular cell
  classes serve?
\item What are the tradeoffs between long-range recurrent excitation
  and inhibition for stimulus detection and discriminability?
\item Does the connectivity in the visual cortex store the visual
  statistics of the environment and how can it begin using this
  information to improve neural coding and low-level inference?
\end{itemize}

Through an iterative process of analysing the model and incorporating
increasing anatomical and physiological detail in the model I make
several concrete predictions about the role in both developmental and
behavioral processes. Firstly I establish that we can create a model
which agrees with anatomical connectivity data, electrophysiological
measurements and the known spatial scales of orientation maps in V1.
Having established this I show that the fast, quickly spiking response
properties of parvalbumin-expressing neurons lead to the development
of broad orientation tuning, which is crucial for robust development
of cortical feature maps. Further, I show that the facilitating and
slow response profile of somatostatin-immunoreactive neurons give rise
to narrower orientation tuning providing orientation dependent
surround suppression. In doing so the thesis will explore the tension
between strong long-range recurrent excitation, useful for detection
under uncertain or weak stimulus conditions and suppression useful to
reduce the redundancy in the neural representation and enhancing
discrimination under strong stimulus conditions. Finally, I show how
the long-range lateral connectivity is able to capture the statistics
of the input, acting to normalize higher-order statistics in the
input, thereby optimizing the neural code, and relate these results to
the known patchy connectivity in the primate visual cortex.

Not only does this thesis outline the first spatially calibrated,
developmental model of the primary visual cortex but it makes specific
predictions about the role of different cell classes, which have been
implicated in a variety of pathologies, most notably in autism,
schizophrenia and Parkinson's. A better understanding of the
developmental disruptions caused by malfunctioning parvalbumin neurons
in particular will provide a better understanding of how to recover
regular function and provides a solid starting point for future
studies looking specifically at these disruptions. Further these
results have clear implications for theoretical neuroscience and
machine learning fields as they clearly predict that not only is the
lateral connectivity in V1 shaped by the visual statistics of the
environment but also suggests that even low-level areas like V1 can
make use of these statistics to optimize neural coding.

\section{Organisation of the Thesis}

After a thorough review of the existing literature, I present a
workflow for easily and reproducibly exploring high-dimensional and
temporally evolving models allowing us to gain insights into
non-linear interactions between different components and parameters in
the model, in ways that were not possible with previous tools.

The following chapter then introduces the initial developmental model
of the visual cortex, summarize what is known about the spatial scales
of different types of connections in primary visual cortex and
establish a correspondence between the model and experimental data. In
the process we present various protocols to characterize the spatial
tuning in the response of the model and various analyses applied to
the connectivity in the model to confirm that model and experiment are
indeed in close agreement.

In the second results chapter I extend this analysis to models of
cortical development and function that takes into account the
anatomical and physiological properties of different cell classes and
begin elucidating their role in experimentally observed phenomena
ranging from self-organization to particular computations performed by
the fully developed cortex, including surround modulation, divisive
gain control and sparse coding.

In the final results chapter I demonstrate how a model that respects
the known properties of different cell classes and their interactions
can develop realistic connectivity patterns that not only allow for
the resulting circuit to robustly respond to highly variable input but
also encode the statistics of the input to aid in the various complex
computations performed by the cortex.

By the end of the thesis I will have presented a unifying account on
how different cell classes in the visual cortex self-organize into a
robust circuit that can encode the natural image statistics of its
rearing environment and make use of that information to improve how
the cortex extracts, encodes and uses information from its
environment.
