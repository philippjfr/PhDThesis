%!TEX root = ../PhDthesis.tex
\chapter{Spatially calibrating models of primary visual cortex}

One of the major obstacles in modern neuroscience is integrating the
vast amount of experimental data that has been generated, highlighting
where different sources of evidence is and isn't in agreement and
offering meaningful hypothesis to resolve discrepancies between
different results. The literature describing the primary visual cortex
is large we have seen that it spans a wide range of levels of
description, from development, circuits and anatomy to surround
modulation, behavioural studies and theoretical models of
computation. In order to provide a better account on how all this
information fits together in overall model describing the organization
and computations performed by the cortex, a unified reference frame
regarding the various spatial scales and their origins is desparately
needed. In particular it has been shown that particularly in the
surround modulation literature known effects are hugely dependent on
stimulus parameters. Here we will present a model that takes these
various levels of evidence into account to allow comparing whether
using known anatomical properties we can predict the known response
properties of the cortex after development. This will allow bridging
between known measurements of anatomy and circuitry and
electrophysiological or even behavioral experiments performed on
visual cortex.

So far only very few attempts have been made at developing models that
take into account the various spatial properties that have been
described in the literature ranging from anatomy to
electrophysiological measurements. In particular to begin making sense
of the surround modulation literature, which is highly dependent on
the precise choice of stimulation protocol, it is essential to take
into the various spatial scales involved. Therefore this chapter will
demonstrate how existing models of cortical development, specifically
the Gain Control Adapation Lateral model (GCAL) \citep{Stevens2013}
can be calibrated to match known measurements of spatial extents more
closely in a new S-patially CAL-ibrated (SCAL) model.

We will begin by looking at the size of receptive fields in the
lateral geniculate nucleus and then progress toward V1, going back and
forth between anatomical and electrophysiological measurements both in
the literature and in the model. Additionally we will pay attention to
the difference in observed connectivity between excitatory and
inhibitory neurons. In tuning this model we will mostly rely on
results gleaned from studies in macaque monkeys, where available,
supplementing missing data with data from other primates and cats.

\section{Spatially calibrating LGN receptive fields}

In spatially calibrating the spatial properties of LGN receptive
fields we must take into account how they will contribute to the V1
receptive fields. One major issue in accurately modeling the LGN
connectivity is that no detailed anatomical measurements exist
describing the extent of LGN neurons and spatial measurements are
highly dependent on stimulus parameters. In \ref{LGNTuning} we
summarize population estimates from a number of studies, measured by
presenting disk masked sine gratings of varying sizes and fitting the
responses with a Difference of Gaussian model.

\begin{table}
  \centering
  \begin{adjustbox}{width=1\textwidth}
  \begin{tabular}{l | l l l l l l}
    Connection   & Literature            & Species  & Ecc. ($\degree$) & Model & Layer & $R_{c/s}$ \\
    \hline
    LGN Center   & \cite{Sceniak2006}    & macaque  & 2-5  & parvo & - & $median = 0.46 \degree$ $mean = 0.5 \degree$ \\
                 & \cite{Levitt2001}     & macaque  & 0-10 & parvo & - & $0.069 \pm 0.076 \degree$ \\
                 & \cite{Spear1994}      & macaque  & 0-10 & parvo & - & $0.087 \pm 0.046 \degree$ \\
                 & \cite{Bonin2005}      & macaque  & 13.9 & parvo & - & $0.6 \pm 0.4 \degree$\\
                 &                       &          &      &       & / & $0.4 \pm 0.2 \degree$ \\
    \hline
    LGN Surround & \cite{Sceniak2006}    & macaque  & 2-5  & parvo & - &$median = 0.51 \degree$ (0.15-0.85) \\
                 & \cite{Levitt2001}     & macaque  & 0-10 & parvo & - & $0.33 \pm 0.076 \degree$ \\
                 & \cite{Spear1994}      & macaque  & 0-10 & parvo & - & $0.53 \pm 0.39 \degree$ \\
                 & \cite{Bonin2005}      & macaque  & 13.9 & parvo & - & $2.0 \pm 1.1 \degree$\\
                 &                       &          &      &       & / & $1.8 \pm 2.6 \degree$\\

    \hline
  \end{tabular}
  \end{adjustbox}
  \caption{Estimates of LGN neuron spatial tuning properties fitted using Difference of Gaussian models
           with either subtractive or divisive suppressive components.}
  \label{LGNTuning}
\end{table}

DoGr: $R = k_c e^{-\frac{r}{r_c}^2} - k_s e^{-\frac{r}{r_s}^2}$

where $R_0$ is the spontaneous response rate, $K_e$ the
excitatory gain, $K_i$ the inhibitory gain, $a$ the excitatory
space constant and $b$ the inhibitory space constant.

The area summation curves were fitted using the following
integrated Difference of Gaussians model:

\begin{equation}
R(s) = R_0 + K_e \int \int re^{-\frac{r^2}{a}} \,
\mathrm{d}r\mathrm{d}\theta - K_i \int\int re^{-\frac{r^2}{b}} \,
\mathrm{d}r\mathrm{d}\theta
\label{iDoG}
\end{equation}

\noindent where $R_0$ is the spontaneous response rate, $K_e$ the
excitatory gain, $K_i$ the inhibitory gain, $a$ the excitatory space
constant and $b$ the inhibitory space constant. The spatial frequency
tuning curve on the other hand was fit with the following function,
which represents a standard Difference of Gaussian model:

\begin{equation}
R = R_0 + R_e - R_i
\label{DoG1}
\end{equation}

\begin{equation}
R = R_0 + \frac{R_e}{1+R_i}
\label{DoG2}
\end{equation}



|  * Spend one day redoing the spatial extents table as discussed in 
|  previous meeting summary
|  * Spend one day redoing the model diagrams as discussed in previous 
|  meeting summary
