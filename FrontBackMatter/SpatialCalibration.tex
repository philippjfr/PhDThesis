%!TEX root = ../PhDthesis.tex
\chapter{Spatially calibrating models of primary visual cortex}

The Gain Control Adapation Lateral model (GCAL) has been able to
demonstrate robust, yet adaptive development of orientation maps,
largely due to the mechanisms it is named after. However, although it
is a good model of map development, matching the time course of
selectivity and stability measurements in ferret V1 (see Stevens et
al. 2013), it was never developed to be spatially calibrated or
closely match anatomy.

In this notebook we will explore the literature on the spatial scales
of connectivity in the early visual system and then adjust the
connectivity in the GCAL model to be consistent with our anatomical
and electrophysiological knowledge about macaque V1. Where macaque
data is not available some of the estimates will be taken from cat
primary visual cortex.

We will begin by looking at the size of receptive fields in the
lateral geniculate nucleus and then progress toward V1, looking at
feedforward and intracortical connectivity. Additionally we will pay
attention to the difference in observed connectivity between
excitatory and inhibitory neurons, which will further highlight how
difficult it is to find correspondence between the GCAL model, with
only one cell class and the real anatomy.

\section{Spatially calibrating LGN receptive fields}

The results from Sceniak et al. (2006) represent the best estimates of
the spatial properties of LGN receptive fields in macaque as they
employ more a more refined measurement protocol than older studies
(full discussion can be found in my First Year Review). The estimated
excitatory extents in this study are significantly larger than
previous estimates, while the suppressive surround estimates are
fairly consistent. Furthermore, the spatial extent of the excitatory
CRF centers were found to be contrast invariant, while both the ECRF
and CRF suppressive surround extents were found to increase at lower
contrast levels. In summary, looking back at all the studies
considered here excitatory CRF extents are generally distributed
between 0.05-0.5 $\deg$ in radius, while inhibitory CRF and
suppressive ECRF radii are distributed distributed anywhere between
0.6-1.5 $\deg$ and the suppression index is quite high (SI>0.8) for
80\% cells.

The spatial profile of LGN Receptive Fields is usually estimated by measuring the response of LGN neurons to moving bar stimuli with varying extents and spatial frequencies. The resulting size tuning curves are then fit using a Difference of Gaussian (DoG) model, expressed as a function of either the spatial frequency (DoGf) or the radius (DoGr).

DoGr: $\(R = k_c e^{-\frac{r}{r_c}^2} - k_s e^{-\frac{r}{r_s}^2}\)$

DoGf: $\(R = k_c e^{-\frac{f}{f_c}^2} - k_s e^{-\frac{f}{f_s}^2}\)$

More recent papers have also attempted to fit the size response curves with integrated Difference of Gaussian models

iDoG $\(R(s) = R_0 + K_e \int \int re^{-\frac{r^2}{a}} \, \mathrm{d}r\mathrm{d}\theta - K_i \int\int re^{-\frac{r^2}{b}} \, \mathrm{d}r\mathrm{d}\theta\)$

where $\(R_0\)$ is the spontaneous response rate, $\(K_e\)$ the excitatory gain, $\(K_i\)$ the inhibitory gain, $\(a\)$ the excitatory space constant and $\(b\)$ the inhibitory space constant.

Using this methodology Sceniak et al. 2006 were able to fit excitatory and inhibitory space constants to a sample of 136 thalamocortical afferents in macaque V1.
