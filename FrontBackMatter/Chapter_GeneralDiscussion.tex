%!TEX root = ../PhDthesis.tex
\chapter{General Discussion}

In this thesis we presented a series of models of the primary visual
cortex to integrate our understanding of how different cell classes
and their synaptic connections can give rise to a robust, yet dynamic
circuit to extract and encode visual features and statistics of the
world. While these models provide valuable contributions and have
allowed us to make a variety of predictions about the role of
inhibitory cell classes in development and contextual modulation, they
also simplify the full complexity in a number of crucial ways. These
simplifications have important implications about the interpretation
of the results but also offer opportunities to extend our modeling
framework.

In this discussion we will explore some of the implications of the
predictions made in the results chapters and place them into context
given our understanding of the underlying circuits. In doing so we
will.


\subsection{Spatial tuning diversity}

* Tuning diversity is limited
* Cannot capture full image statistics
* Variability in responses could affects


\subsection{Sparsity of connections}

* Afferent connections more stable
* Lateral connections more focused (+ and stable)
* Stronger surround modulation effects

\subsection{Interactions between inhibitory cells}

* PV and Sst suppress each other, Sst can disinhibit

\subsection{Complex cells}

Outline the issues with modeling only simple cells and offer
suggestions on extending the model to complex cells

* No more phase disruptions
* Loss of phase specificity for lateral connections
* Inhibition in different circuits

\subsection{Layer specific circuits}

* Clutch cells and push-pull
* PV cells for gain control
* SOM cells for spatial integration
* ViP cell modulation of lateral connections

\subsection{Neuromodulation of visuo-cortical information processing}

One of the original aims when developing the models that were
presented as part of this thesis was to develop a circuit, which would
allow for realistic modulation of specific neural subtypes and
connections. In particular the models are well suited to investigate
how the circuit can be dynamically modulated for task specific
demands, e.g. to model attentional modulation effects.

Specifically the existence of individual populations of inhibitory
neurons and a real long-range excitatory network could be used to
investigate how different neuromodulatory inputs to the cortex affect
the balance between feedforward and recurrent processing. In
particular it is hypothesized that cholinergic and adrenergic
neuromodulators are associated with attentional modulation.

\subsubsection{Cholinergic modulation}

One particular target of interest for further investigation is the
role of cholinergic inputs from the basal forebrain in mediating
attentional effects. Currently there are two main competing theories
of how cholinergic inputs to the cortex alters circuit dynamics,
outlined in the review by \cite{Thiele2013}. The first has been around
for at least two decades and suggests that cholinergic release
increases the feedforward drive and indiscriminately dis- rupts
intracortical processing. Recent evidence on the other seems to
contradict this hypothesis as a gross simplification pointing out that
different intracortical circuits are affected heterogeneously,
suggesting that while feedforward drive may be increased, layer 4
spiny cell firing is decreased, leading to weak inputs being filtered
out, while Sst-ir cell firing is also up-regulated indicating some
similarity between cholinergic modulation and the regular processing
taking place within the circuit under high contrast.

In order to test these hypothesis they could be be put to the test
through specific and well defined manipulations to the LESPI
model. The most basic manipulation would involve increasing
thalamocortical drive and reducing the strength of lateral excitatory
connections in the cortex decreased, shifting the balance of the
circuit towards feedforward processing. Additionally more specific
modulations could be applied by suppressing the Pv-ir neurons and the
Sst-ir neuron activation function steepened. Overall this should shift
the circuit towards a regime akin to the processing that occurs under
high-contrast, reducing spatial and contextual integration in favor of
more accurately representing feedforward input \citep{Roberts2005,
  Roberts2008a, Roberts2008b}.

These types of manipulations offer the chance to test various
hypothesis about how the cortical circuit can be dynamically modulated
to serve a specific task and because could even be used to model the
effect of attention on various perceptual tasks including orientation
discrimination and noisy distractor tasks.
