%!TEX root = ../PhDthesis.tex
\chapter{General Discussion}

In this thesis we presented a series of models of the primary visual
cortex to integrate our understanding of how different cell classes
and their synaptic connections can give rise to a robust, yet dynamic
circuit to extract and encode visual features and statistics of the
world and replicate a wide range of contextual modulation
phenomena. Additionally we presented a workflow for analyzing complex,
high-dimensional datasets easily and reproducibly. Finally we have
presented a wide range of analyses that can be applied to this class
of developmental models to gain a better understanding of the
self-organization processes and to understand how the circuit gives
rise to robust development, contextual modulation and encode the
statistics of the visual environment.

The framework of models and analyses presented here provides huge
scope for future investigations to further explore the role of
different cell classes, neuromodulation, feedback and the development
of lateral connections. In this chapter we will begin by discussing
the implications of the predictions made by the current set of models
and then focus on the different in which the presented models fall
short of capturing the full complexity of the visual circuit and then
suggest specific extensions that will further extend our understanding
of the visual cortex.

One of the major goals set out in this thesis was to begin integrating
across the various levels of description that David Marr laid out in
his seminal work on vision. While individual models describe
individual phenomena very well and we have gathered a huge data about
the cortex from increasingly complex and precise experimental studies,
the theoretical frameworks of computation in the cortex has for the
large part been unchanged. This stands in stark contrast to the field
of machine learning, which recently has made huge advances in the area
of deep neural networks, based losely on the architecture of the
cortex. The models developed as part of this model closely resemble
the architecture of these networks but also model the complexities of
the cortical circuit to far greater detail.

On this basis we have proposed a series of models, which let us relate
the process of learning to the computational functions in the cortex
but also mechanistically explain how different cell classes and
synaptic connections can interact at an algorithmic and
implementational level to give rise to a developed circuit, which
exhibits many of the features observed in the visual cortex. It is one
of the first models that bridges the gaps between low level circuit
interactions with development and learning, giving rise to a developed
circuit that exhibits many of the functional properties of the cortex.

\section{A reproducible workflow for the analysis of complex datasets}

As part of this thesis we have described a general workflow for the
analysis of complex datasets using the HoloViews and Lancet Python
libraries. Every single non-diagramatic plot in this thesis has been
generated using the HoloViews library and is available with a
completely reproducible record of how the figures were generated
available at thesis.philippjfr.com. These tools in themselves provide
a major contribution to greatly improve the productivity of
researchers while ensuring, almost as a byproduct, that each stage of
the research is completely reproducible. This starts with launching
instances of a model either locally or on a cluster, monitoring and
collecting those results and finally generating complex figures and
analyses with minimal amounts of code. The HoloViews library in
particular is now a popular open-source project and is in use in a
wide range of scientific disciplines, has been used in publications
and for teaching materials. As a core designer and developer of the
library, this work represents one of the most fundamental
contributions of this thesis and will improve productivity and
reproducibility far beyond the field of neuroscience.

\section{Spatial Calibration}

One of the first steps in approaching this hugely complex problem was
to adapt existing developmental models in such a way that they would
more closely reflect the real interactions between cell classes and
synaptic connections in the cortex. Therefore the SCAL model was
introduced with spatially calibrated extents allowing us to relate
known anatomical measurements directly to the model and replicate
various measurements protocols, which are highly dependent on the
spatial profiles of connections to the distances in the
model. Additionally we switched from subtractive to divisive
inhibition, as divisive contrast gain-control has long been identified
as a canonical computation in the cortex \citep{Carandini2012}. These
small modification were sufficient to account for contrast dependent
size tuning shifts, suggesting that the shift toward larger size
preferences at low contrasts could almost completely be accounted for
by changes in the gain of the center and surround mechanisms as
suggested by \cite{Cavanaugh2002}. Additionally the spatially
calibrated model provides a foundation to study effects that depend on
the precise spatial configuration of the stimuli.

Furthermore through careful evaluation of the literature we have shown
that the extent of a certain classes of inhibitory neurons is
sufficient to account for the effective Mexican hat connectivity in
the model, that is required for robust self-organization into an
orientation map, while even longer range excitatory projections
provide the true long-range interactions in the circuit.

While the final model presented matched the average size tuning in
parafoveal regions of macaque V1 very well it did not capture the full
diversity in responses. Future modeling work should address this
shortcoming by incorporating the full diversity of spatial tuning
filters and receptive field sizes in the model.

\section{Roles of inhibitory neurons}

The next step was to separate excitatory and inhibitory neural
populations in order to study how different inhibitory cell classes
are suited towards particular functions in the cortex. In doing so we
introduced one of the first developmental models that allows
independently manipulating excitatory and inhibitory cell classes and
were able to make specific predictions about the role of PV and Sst
expressing interneurons in development and surround modulation.

\subsection{Parvalbumin expressing interneurons}

Specifically we discovered that the known properties of PV neurons
make this class of large, feedforward and recurrently driven
interneurons is particularly well suited towards driving the
organization of V1 into a smoothly varying orientation maps. By
changing the properties of the inhibition we were also able to show
that sub- or supra-linear integration by this cell class could result
in disruptions to the development, which may reflect the disruptions
to visual processing that have been observed in pathologies as diverse
as autism, schizophrenia and Parkinsons.

Our analysis also suggested that the feedforward driven PV neurons are
perfectly positioned to provide feedforward inhibition gating how much
afferent drive arrives in the cortex thereby controlling the effective
contrast gain of the circuit. While a model cannot directly confirm
that PV+ neurons serve the same functions in macaque visual cortex as
has been suggested in mouse cortex, the analysis does demonstrate that
feedforward, gain control is an essential property to account for
robust and stable organization and contrast dependent size tuning
shifts. On that basis we predict that PV interneurons in higher
mammals mediate global, feedforward driven contrast gain control as
suggested in a number of studies in mouse V1 \citep{Ma2010,
  Atallah2012, Wilson2012, Nienborg2013}.

At the same time the PV-immunoreactive interneurons is a very broad
classification, which may exhibit considerable
heterogeneity. Specifically clutch cells, which are primarily enriched
in layer 4 likely have very different properties than large basket
cells, which can cover a large area. Indeed there is some evidence
that clutch cells are involved in providing the pull component in a
push-pull model of orientation selectivity and it would be worth
investigating whether this heterogeneity could be incorporated into
the model. This may also explain the ongoing debate about the extent
to which this cell class is tuned for orientation, with studies in
different layers and species coming to very different conclusions
\citep{Cardin2007, Ma2011, Hofer2011}. Future investigations could
also focus on narrowing down precisely how the tuning properties of
this cell class affect the circuit organization.

The major contribution of this model in the context of the thesis is
in establishing a robust model of orientation map development that is
spatially calibrated and has distinct excitatory and inhibitory
populations. The model produces orientation maps, which match
biological maps very closely \citep{Kaschube2010, Stevens2013b}, with
high stability and selectivity very robustly and provides a framework
to build surround modulation models on. The strong divisive gain
control afforded by the inhibitory population also accounts for a
contrast dependent size tuning shifts which matches experiments both
in diversity and magnitude. As such it provides a solid basis to build
further models on.

\subsection{Somatostatin expressing interneurons}

While the parvalbumin population responds very quickly to afferent
input, surround modulation operates over longer spatial and temporal
scales. The other major contribution of this thesis is a general
cortical circuit to mediate long-range interactions via slowly
integrating long-range connections, which mediate both direct
facilitation and di-synaptic suppression through a secondary
interneuron population. After a detailed review of the literature we
concluded that the Sst population, which receive extensive recurrent
inputs from lateral and feedback connections in layer 2/3 integrate
over long spatial scales \citep{Xu2009, Adesnik2012, Nienborg2013}
provide an excellent candidate to mediate long-range and feature
specific surround suppression. Additionally we have shown that an
accelerating response function on the lateral connections between the
excitatory cells and this inhibitory population is sufficient to
account for the contrast dependent changes from surround suppression
to facilitation that have been well characterized in the brain
\citep{Levitt1997, Polat1998, Dragoi2000, Wang2009}.

The full proposed circuit therefore accounts for a huge range of
response properties, developmental processes and surround modulation
effects using a fairly small set of mechanisms and can be used to
inform future experiments. In particular our the modeling suggests a
strong role for a secondary laterally driven inhibitory interneuron
population to provide long-range surround suppression. Whether
feedforward and feedback inhibition can be directly mapped onto the
PV- and Sst- expressing inhibitory cell classes will have to be
experimentally validated however.

\section{Relationship to other surround modulation models}


\section{Roles of lateral connectivity}

In the model the inhibitory populations themselves do not exhibit any
long-range or feature specific connectivity patterns. Long-range
interactions are predominantly driven by long-range excitatory
connections spanning several visual degrees and across multiple
hypercolumns as has been extensively established in the literature
\citep{Weliky1995, Bair2003}. Furhermore detailed analysis of the
connectivity patterns of these long-range connections established not
only that they link iso-orientation columns but also that they are
able to capture the co-occurrence statistics of both simple synthethic
stimuli and more complex natural images. On that basis we suggest that
the long-range patchy connections in layer 2/3 of the cortex merely
reflect the co-linearity that is present in natural images and that
they also capture more subtle properties of natural image statistics
such as co-linearity and other Gestalt rules of visual perception and
are directly responsible for mediating contextual effects such as
contour facilitation, iso-orientation suppression and pop-out.

In order to understand what the lateral connections were actually
representing we developed a number of analyses based on existing
methods for analyzing the co-occurrence statistics in natural images
established by \cite{Geisler2001} and \cite{Perrinet2015}. By adapting
these analyses to extract the co-occurrence statistics from the weight
matrices in the model we were able to conclusively demonstrate that
lateral connections could capture the co-occurrence statistics of the
visual environment. On this basis we predict that lateral connections
across the cortex capture higher-order statistics of the inputs,
providing a statistical model of the relationship between lower level
feature detectors.

Based on these results we predict that the rearing environment of
animals raised in a laboratory environment can strongly affect the
organization of lateral connections in the cortex, which could explain
species differences in the anisotropy of lateral connectivity. To test
this possibility directly the model was trained on both image datasets
drawn from nature and from treeshrew cages and were able to show
highly significant differences in the anisotropy of the
connectivity. This suggests that the differences in lateral connection
isotropy between treeshrews \citep{Bosking1997} and macaques
\citep{Angelucci2002} could at least partially be explained by the
differences in rearing environments of the animals. Experimentalists
should test for this possibility explicitly because it could
significantly influence the properties of surround modulation.

Establishing to what extent these connections are involved in higher
order processing is so far unclear. However based on the results from
the model we can draw several specific conclusions. First of all their
spatial scale is sufficient to capture interactions beyond the size of
the receptive field of most neurons, suggesting they are at least
involved in processing contextual influences from the extra-classical
surround. The extent to which they are involved in these processes
probably depends on the species and cortical eccentricity, as
receptive field sizes vary considerably across species and at least in
some species they can cover a significant portion of the visual space
\citep{Bosking1997}. However they are neither fast nor extensive
enough to account for the full range of surround effects
\citep{Bair2003}.

In this thesis we have demonstrated that these connections can mediate
surround modulation effects and that they represent the statistics of
the input. However future analyses should establish to what extent
small differences in the statistics of the connections can affect
surround modulation and the sparsity of representation. Now that the
statistics embedded in the connections can be easily extracted it
should be possible to establish how modulating the statistics of the
inputs will affect surround modulation and the sparse coding of visual
inputs.

\subsection{Interactions between cell classes}

In the current model interactions between the inhibitory cell classes
were largely ignored. However based on experiments it is thought that
PV and Sst neurons strongly inhibit each other. Through some
exploratory work it was observed that Sst suppression of PV neurons
could lead to interesting dynamics where they would essentially
disinhibit the local excitatory population, reducing the amount of
contrast gain-control and therefore reducing the effective integration
area. This is a very interesting avenue to pursue as this could
effectively model some of the features of visual attention, where
feedforward drive is increased relative to the lateral processing,
effectively mirroring the effects of increasing the contrast.

\section{Sparsity of connections}

One major avenue of work that was developed for this thesis but
ultimately was not used was developing a sparse matrix implementation
of the core algorithms used to simulate the model. This work has since
been extended to work on GPUs enabling significant speedups in
simulating large models. This work will not only allow for the
implementation of much larger models but also make the existing models
more realistic. Specifically it is known that V1 neurons in layer 4
receive afferent input from just a small number of afferent
projections and lateral connections are highly sparse in their pattern
of innvervation.

Simulating sparsity could therefore improve the realism of the models
and at the same time improve a variety of properties in the
model. Specifically initial exploration of sprouting and retraction
algorithms for controlling the sparsity of synaptic connections
demonstrated that orientation map development could proceed with just
a small fraction of the connections that are simulated in the dense
version and the resulting models were considerably more stable because
the afferent weights could not easily move around. This provides more
realism for the simulation of orientation map development.

More interestingly however letting the lateral connections develop
sparsely could match the known timecourse of the emergence of patchy
connections in the cortex very well \citep{Ruthazer1996}, emerging in
concert with the orientation map and then slowly clustering in
increasingly distant iso-orientation patches. This would also remove
the need to threshold the weights after development achieving the
sparse distribution of connections required for strong surround
modulation without an artificial mechanism.

\section{Complex cells}

One major limitation of models presented in this thesis is that they
are purely a model of simple cells, even though long-range connections
and surround modulation effects are generally ascribed to layer 2/3,
which has both simple and complex cells. This may have considerable

Outline the issues with modeling only simple cells and offer
suggestions on extending the model to complex cells

* No more phase disruptions
* Loss of phase specificity for lateral connections
* Inhibition in different circuits

\section{Temporal responses}


\section{Neuromodulation of visuo-cortical information processing}

One of the original aims when developing the models that were
presented as part of this thesis was to develop a circuit, which would
allow for realistic modulation of specific neural subtypes and
connections. In particular the models are well suited to investigate
how the circuit can be dynamically modulated for task specific
demands, e.g. to model attentional modulation effects.

Specifically the existence of individual populations of inhibitory
neurons and a real long-range excitatory network could be used to
investigate how different neuromodulatory inputs to the cortex affect
the balance between feedforward and recurrent processing. In
particular it is hypothesized that cholinergic and adrenergic
neuromodulators are associated with attentional modulation.

\subsection{Cortical feedback}


\subsection{Cholinergic modulation}

One particular target of interest for further investigation is the
role of cholinergic inputs from the basal forebrain in mediating
attentional effects. Currently there are two main competing theories
of how cholinergic inputs to the cortex alters circuit dynamics,
outlined in the review by \cite{Thiele2013}. The first has been around
for at least two decades and suggests that cholinergic release
increases the feedforward drive and indiscriminately disrupts
intracortical processing. Recent evidence on the other seems to
contradict this hypothesis as a gross simplification pointing out that
different intracortical circuits are affected heterogeneously,
suggesting that while feedforward drive may be increased, layer 4
spiny cell firing is decreased, leading to weak inputs being filtered
out, while Sst-ir cell firing is also up-regulated indicating some
similarity between cholinergic modulation and the regular processing
taking place within the circuit under high contrast.

In order to test these hypothesis they could be be put to the test
through specific and well defined manipulations to the LESPI
model. The most basic manipulation would involve increasing
thalamocortical drive and reducing the strength of lateral excitatory
connections in the cortex, shifting the balance of the circuit towards
feedforward processing. Additionally more specific modulations could
be applied by suppressing the Pv-ir neurons and the Sst-ir neuron
activation function steepened. Overall this should shift the circuit
towards a regime akin to the processing that occurs under
high-contrast, reducing spatial and contextual integration in favor of
more accurately representing feedforward input \citep{Roberts2005,
  Roberts2008a, Roberts2008b}.

These types of manipulations offer the chance to test various
hypothesis about how the cortical circuit can be dynamically modulated
to serve a specific task and because could even be used to model the
effect of attention on various perceptual tasks including orientation
discrimination and noisy distractor tasks.
