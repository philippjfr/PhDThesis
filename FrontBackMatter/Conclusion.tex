%!TEX root = ../PhDthesis.tex
\chapter{Conclusion}

The overarching goal in computational and theoretical neuroscience has
always been providing a comprehensive theory of cortical
computation. As we have discussed throughout this thesis, this goal has
seemingly not moved much closer over the past decades, despite the
tremendous amount of data that has been collected over this time and
more recently the huge success of deep learning models in achieving
performance that rivals humans in very narrowly defined domains. The
main contribution of this thesis is in providing a mechanistic account
of how even early sensory areas such as the primary visual cortex can
accurately extract and encode higher order statistical relationships,
and can then make use of these statistics to dynamically modulate neural
responses depending on the context and task-specific demands. We
suggest that the ability of cortical areas to encode the dependencies
between its inputs is a fundamental property that is shared across
cortical areas, allowing the brain to dynamically adapt to contextual
and task-specific demands.
